% =====================================================================
% Archivo: EjemploFiguraHorizontal.tex
% Propósito: Demostración de figura horizontal con clase cel.cls
% Integración: Guía horizontal SENER
% =====================================================================

\documentclass{cel}

% --- Metadatos del Documento ---
\title{Ejemplo de Figura Horizontal}
\subtitle{Demostración de Modo Landscape}
\author{SENER}
\date{\today}
\institucion{Secretaría de Energía (SENER)}
\unidad{Unidad de Planeación Energética}
\setDocumentoCorto{Ejemplo Horizontal}
\palabrasclave{Figura, Horizontal, Landscape}
\version{1.0}

\begin{document}

\maketitle

\section{Introducción}

Este documento demuestra el uso de figuras en modo horizontal utilizando la clase \texttt{cel.cls} y siguiendo las mejores prácticas de la Guía de Modo Horizontal SENER.

\section{Ejemplo de Figura Vertical Normal}

Primero mostramos una figura en modo vertical normal:

\begin{figure}[H]
    \caption{Ejemplo de figura vertical normal}
    \centering
    \rule{10cm}{6cm} % Rectángulo de ejemplo
    \label{fig:vertical}
\end{figure}

\fuente{Elaboración propia con datos de ejemplo}

\section{Transición a Modo Horizontal}

A continuación se presenta una figura en modo horizontal que aprovecha todo el espacio disponible de la página.

% Ejemplo de figura horizontal
\begin{figuraespecial}
    \seccionHorizontal{Balance Energético Nacional}
    
    \captionHorizontal{Distribución de la Generación de Energía Eléctrica por Tecnología en México (2024)}
    
    % Crear una figura de ejemplo más grande para modo horizontal
    \imagenHorizontal{img/ejemplo-horizontal.png}{fig:balance-horizontal}
    
    \fuenteHorizontal{Elaboración SENER con datos del Sistema de Información Energética (SIE) y CENACE\footnotemark}
    \footnotetext{Los datos incluyen generación de centrales públicas y privadas conectadas al Sistema Eléctrico Nacional}
\end{figuraespecial}

\section{Continuación en Modo Vertical}

Después de la figura horizontal, el documento continúa en modo vertical normal. La transición es automática y mantiene la consistencia del formato.

\subsection{Beneficios del Modo Horizontal}

\begin{itemize}
    \item \textbf{Mayor espacio disponible}: Aprovecha el ancho completo de la página
    \item \textbf{Mejor visualización}: Ideal para gráficos complejos y tablas extensas
    \item \textbf{Identidad institucional}: Mantiene los elementos de marca SENER
    \item \textbf{Transición fluida}: Cambio automático entre modos vertical y horizontal
\end{itemize}

\subsection{Elementos Incluidos en Modo Horizontal}

El modo horizontal incluye automáticamente:

\begin{enumerate}
    \item Línea dorada institucional en la parte superior
    \item Logo del Gobierno de México rotado
    \item Numeración de página en posición correcta
    \item Nombre corto del documento
    \item Línea de pie de página
    \item Soporte para notas al pie
\end{enumerate}

\section{Ejemplo de Tabla Horizontal}

También es posible crear tablas en modo horizontal:

\begin{tablaespecial}
    \tituloHorizontal{Matriz de Comparación de Tecnologías}
    
    \begin{tabladoradoLargo}
        \tiny
        \begin{xltabular}{\anchoHorizontal}{|p{3cm}|p{2.5cm}|p{2.5cm}|p{2.5cm}|p{2.5cm}|p{2.5cm}|p{2.5cm}|}
        \caption{Comparación de Tecnologías de Generación Eléctrica} \\
        \toprule
        \rowcolor{gobmxDorado}
        \encabezadodorado{Tecnología} & 
        \encabezadodorado{Capacidad (MW)} & 
        \encabezadodorado{Factor de Planta (\%)} & 
        \encabezadodorado{Costo LCOE (USD/MWh)} & 
        \encabezadodorado{Emisiones (kg CO₂/MWh)} & 
        \encabezadodorado{Vida Útil (años)} & 
        \encabezadodorado{Clasificación CEL} \\
        \midrule
        \endhead
        
        \textbf{Solar Fotovoltaica} & 15,000 & 25 & 45 & 0 & 25 & Limpia \\
        \hline
        \textbf{Eólica} & 12,500 & 35 & 38 & 0 & 20 & Limpia \\
        \hline
        \textbf{Hidroeléctrica} & 8,200 & 45 & 42 & 0 & 50 & Limpia \\
        \hline
        \textbf{Geotérmica} & 1,200 & 75 & 55 & 0 & 30 & Limpia \\
        \hline
        \textbf{Ciclo Combinado} & 25,000 & 60 & 65 & 350 & 25 & Convencional \\
        \hline
        \textbf{Carboeléctrica} & 5,500 & 70 & 75 & 820 & 40 & Convencional \\
        \bottomrule
        \end{xltabular}
    \end{tabladoradoLargo}
    
    \fuenteHorizontal{Elaboración SENER con datos de PRODESEN 2024-2038 y reportes de la CNE}
\end{tablaespecial}

\section{Conclusiones}

El modo horizontal de la clase \texttt{cel.cls} proporciona una herramienta poderosa para la presentación de información compleja manteniendo la identidad institucional de SENER. Su implementación automática facilita la creación de documentos profesionales sin comprometer la funcionalidad.

\end{document}