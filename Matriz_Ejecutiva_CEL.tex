% =====================================================================
% Archivo: Matriz_Ejecutiva_CEL.tex
% Propósito: Matriz ejecutiva de áreas de oportunidad del Sistema CEL
% Destinatario: Nivel directivo SENER
% Estilo: Español institucional (México) con clase cel.cls
% =====================================================================

\documentclass{cel}

% --- Metadatos del Documento ---
\title{Matriz de Áreas de Oportunidad del Sistema CEL}
\subtitle{Resumen Ejecutivo para Nivel Directivo}
\author{SENER}
\date{\today}
\institucion{Secretaría de Energía (SENER)}
\unidad{Subsecretaría de Planeación y Transición Energética}
\setDocumentoCorto{Matriz CEL}
\palabrasclave{CEL, Oportunidades, Sistema, Modernización}
\version{1.0}

% --- Metadatos PDF/UA ---
\hypersetup{
  pdftitle={Matriz de Áreas de Oportunidad del Sistema CEL},
  pdfauthor={SENER},
  pdfsubject={Sistema CEL - Áreas de Oportunidad},
  pdfkeywords={CEL, Oportunidades, Sistema, Modernización},
  pdfversion={1}
}

\begin{document}

% Portada de sección estilo CEL
\portadaseccion{MATRIZ}{Áreas de Oportunidad del Sistema CEL}{Enero 2026}

% =====================================================================
% CONTENIDO PRINCIPAL
% =====================================================================

\section*{Matriz de Áreas de Oportunidad del Sistema CEL}
\addcontentsline{toc}{section}{Matriz de Áreas de Oportunidad}

\subsection*{Contexto}

La Subsecretaría de Planeación y Transición Energética se encuentra desarrollando un análisis integral y detallado de las áreas de oportunidad del Sistema de Certificados de Energía Limpia (S-CEL). El presente documento presenta una matriz ejecutiva de los principales hallazgos identificados para consideración del nivel directivo.

\subsection*{Matriz Principal de Hallazgos}

\begin{tabladoradoLargo}
    \tiny
    \begin{xltabular}{\textwidth}{|p{2cm}|p{3cm}|p{4cm}|p{3.5cm}|p{1.5cm}|}
    \caption{Matriz de Áreas de Oportunidad del Sistema CEL} \label{tab:matriz_oportunidades} \\
    \toprule
    \rowcolor{gobmxDorado} 
    \encabezadodorado{Bloque} & 
    \encabezadodorado{Tema} & 
    \encabezadodorado{Problemática} & 
    \encabezadodorado{Propuesta de Mejora} & 
    \encabezadodorado{Prioridad} \\
    \midrule
    \endhead
    
    \textbf{I. Entrada al Sistema} & 
    Proceso de inscripción & 
    Complejidad documental excesiva y plazos indefinidos & 
    Digitalizar proceso con plazos máximos de 30 días hábiles & 
    \textbf{CRÍTICA} \\
    \hline
    
    \textbf{I. Entrada al Sistema} & 
    Registro y administración & 
    Gestión manual propensa a errores y falta de interoperabilidad & 
    Automatizar gestión de cuentas con APIs de integración & 
    \textbf{ALTA} \\
    \hline
    
    \textbf{I. Entrada al Sistema} & 
    Medición y datos & 
    Cobertura parcial de generación distribuida y dependencia de reportes manuales & 
    Implementar sistema universal de medición con validación automática & 
    \textbf{ALTA} \\
    \hline
    
    \textbf{II. Otorgamiento} & 
    Modalidades operativas & 
    Dependencia excesiva de reportes manuales y criterios diferenciados & 
    Automatizar captura de datos y homologar criterios por modalidad & 
    \textbf{CRÍTICA} \\
    \hline
    
    \textbf{II. Otorgamiento} & 
    Dictámenes técnicos & 
    Vigencia limitada sin renovación automática y conflictos de interés & 
    Establecer renovación automática condicionada y reglas de independencia & 
    \textbf{MEDIA} \\
    \hline
    
    \textbf{II. Otorgamiento} & 
    Conteo y fraccionamiento & 
    Ausencia de criterios específicos para redondeo y centrales híbridas & 
    Establecer redondeo a 4 decimales y acumulación separada por tecnología & 
    \textbf{ALTA} \\
    \hline
    
    \textbf{II. Otorgamiento} & 
    Trazabilidad & 
    Desconexión entre CEL y Factor de Emisión SEN & 
    Vincular CEL con impacto ambiental real e implementar verificación criptográfica & 
    \textbf{MEDIA} \\
    \hline
    
    \textbf{III. Disponibilidad} & 
    Balance oferta-demanda & 
    Ausencia de monitoreo en tiempo real y falta de proyecciones sistemáticas & 
    Implementar sistema de monitoreo continuo con alertas tempranas & 
    \textbf{CRÍTICA} \\
    \hline
    
    \textbf{III. Disponibilidad} & 
    Mecanismos de flexibilidad & 
    Criterios limitados de aplicación y falta de transparencia & 
    Ampliar criterios de flexibilidad y mejorar transparencia en aplicación & 
    \textbf{MEDIA} \\
    \hline
    
    \textbf{III. Disponibilidad} & 
    Entidades voluntarias & 
    Definición limitada y proceso sin verificación de adicionalidad & 
    Ampliar definición con objetivos ambientales y verificar adicionalidad & 
    \textbf{MEDIA} \\
    \hline
    
    \textbf{IV. Transacciones} & 
    Mercado de CEL & 
    Dependencia excesiva del mercado centralizado y limitaciones del Boletín & 
    Diversificar mecanismos con plataformas complementarias y matching automático & 
    \textbf{ALTA} \\
    \hline
    
    \textbf{IV. Transacciones} & 
    Transacciones bilaterales & 
    Proceso manual con alta fricción y ausencia de mecanismos de garantía & 
    Automatizar con smart contracts e implementar sistema de clearing & 
    \textbf{ALTA} \\
    \hline
    
    \textbf{IV. Transacciones} & 
    Bolsa No Onerosa & 
    Distorsión del mercado por asignación gratuita y criterios subjetivos & 
    Eliminar asignación gratuita y establecer criterios taxativos & 
    \textbf{CRÍTICA} \\
    \hline
    
    \textbf{V. Precios} & 
    Formación de precios & 
    Ausencia de metodología para precios bilaterales y desconexión ambiental & 
    Establecer metodología de reporte y desarrollar precio de referencia ambiental & 
    \textbf{CRÍTICA} \\
    \hline
    
    \textbf{V. Precios} & 
    Factor de emisiones & 
    Limitaciones del Boletín como descubridor de precios & 
    Mejorar funcionalidades del Boletín e implementar mecanismos de estabilización & 
    \textbf{MEDIA} \\
    \hline
    
    \textbf{VI. Instrumentos} & 
    Contratos de cobertura & 
    Transferencias automáticas sin consideración de condiciones actuales & 
    Actualizar mecanismo de transferencia con validación de condiciones & 
    \textbf{MEDIA} \\
    \hline
    
    \textbf{VI. Instrumentos} & 
    Subastas de CEL & 
    Suspensión indefinida sin criterios de reactivación & 
    Establecer criterios específicos y cronograma para evaluación de reactivación & 
    \textbf{MEDIA} \\
    \hline
    
    \textbf{VII. Cumplimiento} & 
    DECLARACEL & 
    Sistema con limitaciones de eficiencia y transparencia & 
    Modernizar sistema para mayor eficiencia y reducción de errores & 
    \textbf{ALTA} \\
    \hline
    
    \textbf{VII. Cumplimiento} & 
    Régimen de sanciones & 
    Rango sancionador excesivamente amplio sin criterios de graduación & 
    Establecer metodología específica de graduación con criterios objetivos & 
    \textbf{ALTA} \\
    \hline
    
    \textbf{VII. Cumplimiento} & 
    Transparencia & 
    Información limitada y falta de acceso público a datos relevantes & 
    Implementar sistema integral de transparencia con dashboard público & 
    \textbf{ALTA} \\
    \bottomrule
    \end{xltabular}
\end{tabladoradoLargo}

\newpage

\section*{Consideraciones Técnicas}

\begin{glowBox}[gobmxDorado]{Nota Importante: Integración con SNIER}
\textbf{Integración Técnica con el Sistema Nacional de Información Energética y de Recursos (SNIER)}

El Sistema CEL debe integrarse completamente con el SNIER para garantizar coherencia en la información energética nacional y facilitar la toma de decisiones basada en datos integrados.

\textbf{Propuesta de Control Secundario SENER:}
La Subsecretaría de Planeación y Transición Energética debe implementar un sistema de monitoreo a través de una API que replique el ciclo de vida completo de los CEL como control secundario y check del sistema principal de la CNE, permitiendo monitorear en tiempo real:

\begin{itemize}
    \item Otorgamiento de certificados por tecnología y región
    \item Cumplimiento de obligaciones por participante
    \item Aplicación de sanciones y medidas correctivas
    \item Impacto en metas nacionales de energías limpias
    \item Señales de política energética y ajustes necesarios
\end{itemize}

Esta arquitectura de control dual garantizará la supervisión efectiva del instrumento de política energética más importante para la transición energética nacional.
\end{glowBox}

\end{document}