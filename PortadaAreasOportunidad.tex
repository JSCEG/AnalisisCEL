% =====================================================================
% Archivo: PortadaAreasOportunidad.tex
% Propósito: Portada de sección para "Áreas de Oportunidad del Sistema CEL"
% Estilo: Español institucional (México) con clase cel.cls
% =====================================================================

\documentclass{cel}

% --- Metadatos del Documento ---
\title{Portada de Sección - Áreas de Oportunidad CEL}
\subtitle{Demostración de Portada Institucional}
\author{SENER}
\date{\today}
\institucion{Secretaría de Energía (SENER)}
\unidad{Unidad de Planeación Energética}
\setDocumentoCorto{Portada CEL}
\palabrasclave{Portada, CEL, SENER}
\version{1.0}

\begin{document}

% Ejemplo de portada de sección principal para el documento completo
\portadaseccion{}{Áreas de Oportunidad del Sistema de Certificados de Energía Limpia}{Análisis Integral y Propuestas de Modernización para la Transición Energética de México}

% Página de contenido después de la portada
\section*{Introducción al Documento}

Este documento presenta un análisis integral de las áreas de oportunidad identificadas en el Sistema de Certificados de Energía Limpia (S-CEL) de México.

\subsection*{Objetivos del Análisis}

\begin{itemize}
    \item Identificar brechas operativas y normativas en el S-CEL actual
    \item Proponer soluciones sistémicas para la modernización del sistema
    \item Establecer una hoja de ruta para la implementación de mejoras
    \item Fortalecer la soberanía energética nacional
\end{itemize}

\subsection*{Alcance del Documento}

El análisis abarca 7 bloques temáticos principales:

\begin{enumerate}
    \item \textbf{Entrada al Sistema y Base Operativa}
    \item \textbf{Otorgamiento del Certificado}
    \item \textbf{Oferta, Disponibilidad y Flexibilidad}
    \item \textbf{Mecanismos de Transacción}
    \item \textbf{Precio y Señales Ambientales}
    \item \textbf{Instrumentos de Mediano y Largo Plazo}
    \item \textbf{Cumplimiento, Sanción y Transparencia}
\end{enumerate}

% Ejemplo de portadas adicionales para cada bloque
\portadaseccion{I}{Entrada al Sistema y Base Operativa}{Procesos de inscripción, registro y administración del S-CEL}

\section*{Bloque I: Entrada al Sistema}

Este bloque analiza los procesos fundamentales de acceso al S-CEL.

\portadaseccion{II}{Otorgamiento del Certificado}{Modalidades operativas y dictámenes técnicos}

\section*{Bloque II: Otorgamiento}

Análisis de las modalidades de emisión de certificados.

\portadaseccion{III}{Oferta, Disponibilidad y Flexibilidad}{Disponibilidad real de CEL y mecanismos de flexibilidad}

\section*{Bloque III: Oferta y Flexibilidad}

Evaluación del balance oferta-demanda en el mercado de CEL.

\portadaseccion{IV}{Mecanismos de Transacción}{Mercado de CEL y transacciones bilaterales}

\section*{Bloque IV: Transacciones}

Análisis del mercado secundario y mecanismos de intercambio.

\portadaseccion{V}{Precio y Señales Ambientales}{Formación de precios y factor de emisiones}

\section*{Bloque V: Precios}

Estudio de la formación de precios y señales de mercado.

\portadaseccion{VI}{Instrumentos de Mediano y Largo Plazo}{Contratos de cobertura y subastas}

\section*{Bloque VI: Instrumentos de Largo Plazo}

Evaluación de instrumentos de planeación energética.

\portadaseccion{VII}{Cumplimiento, Sanción y Transparencia}{DECLARACEL, sanciones y transparencia}

\section*{Bloque VII: Cumplimiento}

Análisis del régimen de cumplimiento y transparencia.

% Portada final de conclusiones
\portadaseccion{}{Conclusiones y Recomendaciones}{Hoja de Ruta para la Modernización del Sistema CEL}

\section*{Síntesis Ejecutiva}

Las áreas de oportunidad identificadas en este análisis representan una oportunidad histórica para posicionar a México como líder en certificación de energías limpias en América Latina.

\subsection*{Recomendaciones Prioritarias}

\begin{enumerate}
    \item \textbf{Digitalización Integral}: Modernizar la plataforma S-CEL con tecnologías de vanguardia
    \item \textbf{Transparencia de Mercado}: Implementar mecanismos de transparencia en precios y transacciones
    \item \textbf{Interoperabilidad}: Conectar el S-CEL con otros sistemas energéticos nacionales
    \item \textbf{Marco Normativo}: Actualizar las DACG para reflejar las mejores prácticas internacionales
\end{enumerate}

\subsection*{Impacto Esperado}

La implementación de estas propuestas contribuirá significativamente a:

\begin{itemize}
    \item Fortalecer la soberanía energética nacional
    \item Acelerar la transición hacia energías limpias
    \item Mejorar la competitividad del sector energético mexicano
    \item Posicionar a México como referente regional en certificación energética
\end{itemize}

\end{document}