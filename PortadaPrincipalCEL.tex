% =====================================================================
% Archivo: PortadaPrincipalCEL.tex
% Propósito: Portada principal para "Áreas de Oportunidad del Sistema CEL"
% Estilo: Español institucional (México) con clase cel.cls
% Uso: Como sección independiente del documento principal
% =====================================================================

\documentclass{cel}

% --- Metadatos del Documento ---
\title{Áreas de Oportunidad del Sistema de Certificados de Energía Limpia}
\subtitle{Análisis Integral y Propuestas de Modernización}
\author{SENER}
\date{\today}
\institucion{Secretaría de Energía (SENER)}
\unidad{Unidad de Planeación Energética}
\setDocumentoCorto{Áreas CEL}
\palabrasclave{CEL, Modernización, Transición Energética}
\version{1.0}

\begin{document}

% =====================================================================
% PORTADA PRINCIPAL DEL DOCUMENTO
% =====================================================================

% Portada de sección principal - Sin número de bloque
\portadaseccion{}{Áreas de Oportunidad del Sistema de Certificados de Energía Limpia}{Análisis Integral y Propuestas de Modernización para la Transición Energética de México}

% =====================================================================
% CONTENIDO DESPUÉS DE LA PORTADA
% =====================================================================

\section*{Presentación}

El presente documento constituye un análisis exhaustivo de las áreas de oportunidad identificadas en el Sistema de Certificados de Energía Limpia (S-CEL) de México, desarrollado por la Secretaría de Energía en el marco de la Nueva Ley del Sector Eléctrico (2025).

\subsection*{Contexto Institucional}

La reconfiguración del sector energético mexicano, con la creación de la Comisión Nacional de Energía (CNE) y la modernización del marco regulatorio, presenta una oportunidad única para transformar el S-CEL en un instrumento de vanguardia para la transición energética nacional.

\subsection*{Metodología del Análisis}

El análisis se estructura en **siete bloques temáticos** que abordan de manera integral todos los aspectos operativos, normativos y tecnológicos del sistema:

\begin{enumerate}
    \item \textbf{Entrada al Sistema y Base Operativa}
    \item \textbf{Otorgamiento del Certificado}
    \item \textbf{Oferta, Disponibilidad y Flexibilidad}
    \item \textbf{Mecanismos de Transacción}
    \item \textbf{Precio y Señales Ambientales}
    \item \textbf{Instrumentos de Mediano y Largo Plazo}
    \item \textbf{Cumplimiento, Sanción y Transparencia}
\end{enumerate}

\subsection*{Enfoque de Modernización}

Cada bloque temático incluye:

\begin{itemize}
    \item \textbf{Diagnóstico de la situación actual}
    \item \textbf{Identificación de brechas y oportunidades}
    \item \textbf{Propuestas de mejora normativa y operativa}
    \item \textbf{Arquitectura de sistemas modernizados}
    \item \textbf{Beneficios esperados de la implementación}
\end{itemize}

\subsection*{Visión Estratégica}

Las propuestas contenidas en este documento buscan posicionar a México como líder regional en certificación de energías limpias, fortaleciendo la soberanía energética nacional y acelerando la transición hacia un modelo energético más limpio y sustentable.

\newpage

% =====================================================================
% EJEMPLO DE INTEGRACIÓN EN DOCUMENTO PRINCIPAL
% =====================================================================

\section*{Cómo Integrar esta Portada en AreasdeOportunidadCEL.tex}

Para integrar esta portada en tu documento principal, simplemente reemplaza la línea:

\begin{verbatim}
% Portada institucional con fondo personalizado
\portadafondo[img/portada.png]
\end{verbatim}

Por:

\begin{verbatim}
% Portada de sección principal
\portadaseccion{}{Áreas de Oportunidad del Sistema de 
Certificados de Energía Limpia}{Análisis Integral y 
Propuestas de Modernización para la Transición 
Energética de México}
\end{verbatim}

\subsection*{Ventajas de usar \texttt{\\portadaseccion}}

\begin{enumerate}
    \item \textbf{Consistencia Visual}: Mantiene el mismo estilo que las portadas de los bloques temáticos
    \item \textbf{Identidad Institucional}: Incluye automáticamente los elementos de marca SENER
    \item \textbf{Flexibilidad}: Permite personalizar el título y subtítulo fácilmente
    \item \textbf{Integración}: Se integra perfectamente con el resto del documento
\end{enumerate}

\subsection*{Personalización}

Puedes personalizar la portada modificando los parámetros:

\begin{itemize}
    \item \textbf{Primer parámetro}: Número o identificador del bloque (vacío para portada principal)
    \item \textbf{Segundo parámetro}: Título principal
    \item \textbf{Tercer parámetro}: Subtítulo o descripción
\end{itemize}

\end{document}