\DocumentMetadata{
  pdfversion=2.0,
  lang=es-MX,
  pdfstandard=ua-2
}

\documentclass{cel}

% --- Metadatos PDF/UA (Accesibilidad Universal) ---
\hypersetup{
  pdftitle={Resolución de Sanciones CEL - CRE/RES/248/2016},
  pdfauthor={CRE},
  pdfsubject={Sanciones CEL},
  pdfkeywords={Energía, CEL, Sanciones, Regulación},
  pdfversion={1}
}

% --- Metadatos del Documento ---
\title{Resolución de Sanciones CEL}
\subtitle{CRE/RES/248/2016 - Criterios para Imposición de Sanciones}
\author{Comisión Reguladora de Energía}
\date{27 de abril de 2016}
\institucion{Comisión Reguladora de Energía (CRE)}
\unidad{Dirección General de Electricidad}
\setDocumentoCorto{Sanciones CEL}
\palabrasclave{Energía, CEL, Sanciones}
\version{1}

\begin{document}

\portadafondo[img/portadacel.png]

% Índice General (TOC)
{
  \hypersetup{linkcolor=gobmxGuinda}
  \tableofcontents
}
\newpage

\phantomsection
\addcontentsline{toc}{section}{Introducción}
\section*{Introducción}

La presente resolución establece los criterios para la imposición de sanciones que deriven del incumplimiento de las obligaciones en materia de energías limpias, específicamente relacionadas con la adquisición de Certificados de Energías Limpias (CEL).

Publicada en el Diario Oficial de la Federación el 27 de abril de 2016, esta resolución busca proporcionar certeza jurídica y criterios objetivos para la determinación de multas por incumplimiento en la adquisición de CEL.

\section{Datos Generales de la Resolución}

\begin{tabladoradoCorto}
  \caption{Información General de la Resolución}
  \begin{tabularx}{\textwidth}{L{0.30\textwidth} X}
    \toprule
    \rowcolor{gobmxDorado} \encabezadodorado{Elemento} & \encabezadodorado{Detalle} \\
    \midrule
    \textbf{Número de Resolución} & RES/248/2016 \\
    \textbf{Fecha de Publicación} & 27 de abril de 2016 \\
    \textbf{Autoridad Emisora} & Comisión Reguladora de Energía (CRE) \\
    \textbf{Medio de Publicación} & Diario Oficial de la Federación \\
    \textbf{Materia} & Sanciones por incumplimiento de obligaciones CEL \\
    \textbf{Vigencia} & Al día siguiente de su publicación \\
    \bottomrule
  \end{tabularx}
\end{tabladoradoCorto}

\section{Marco Jurídico de Referencia}

\subsection{Fundamentos Constitucionales y Legales}

La resolución se fundamenta en el marco jurídico de la Reforma Energética y las disposiciones específicas en materia de energías limpias:

\textbf{Fundamentos principales:}

\textbf{• Constitución Política}: Artículos 27 y 28 sobre rectoría del Estado en materia energética

\textbf{• Ley de la Industria Eléctrica (LIE)}: Artículos 121-128 sobre obligaciones de energías limpias

\textbf{• Ley de Transición Energética (LTE)}: Disposiciones sobre aprovechamiento sustentable

\textbf{• Ley de Órganos Reguladores}: Facultades de la CRE en materia sancionadora

\subsection{Antecedentes Normativos}

\begin{tabladoradoLargo}
    \tiny
    \begin{xltabular}{\textwidth}{|l|X|X|}
    \caption{Cronología de Instrumentos Normativos Relevantes} \\
    \toprule
    \rowcolor{gobmxDorado} \encabezadodorado{Fecha} & \encabezadodorado{Instrumento} & \encabezadodorado{Contenido Relevante} \\
    \midrule
    \endhead
    
    \textbf{20/12/2013} & Decreto de Reforma Energética & Modificaciones constitucionales en materia energética \\ \hline
    \textbf{11/08/2014} & Ley de la Industria Eléctrica & Marco regulatorio del sector eléctrico \\ \hline
    \textbf{31/10/2014} & Reglamento de la LIE & Disposiciones reglamentarias \\ \hline
    \textbf{31/03/2015} & Aviso SENER & Requisito para adquisición de CEL en 2018 \\ \hline
    \textbf{08/09/2015} & Bases del Mercado Eléctrico & Reglas del Mercado Eléctrico Mayorista \\ \hline
    \textbf{24/12/2015} & Ley de Transición Energética & Aprovechamiento sustentable de energía \\ \hline

    \bottomrule
    \end{xltabular}
\end{tabladoradoLargo}

\section{Definiciones y Conceptos Clave}

\subsection{Certificados de Energías Limpias (CEL)}

De conformidad con el artículo 3 de la LIE, se entiende como Certificado de Energías Limpias al título emitido por la Comisión que acredita la producción de un monto determinado de energía eléctrica a partir de Energías Limpias.

\subsection{Participantes Obligados}

Según el artículo 123 de la LIE, constituyen una obligación para:

\textbf{• Suministradores}

\textbf{• Usuarios Calificados Participantes del Mercado}

\textbf{• Usuarios Finales que reciban energía eléctrica por abasto aislado}

\textbf{• Titulares de contratos de interconexión legados}

\subsection{Mecanismo de Flexibilidad}

El Transitorio Vigésimo Segundo de la LTE establece un mecanismo especial para los primeros cuatro años:

\begin{tabladoradoCorto}
  \caption{Condiciones del Mecanismo de Flexibilidad}
  \begin{tabularx}{\textwidth}{L{0.25\textwidth} X X}
    \toprule
    \rowcolor{gobmxDorado} \encabezadodorado{Condición} & \encabezadodorado{Criterio} & \encabezadodorado{Diferimiento Permitido} \\
    \midrule
    \textbf{Insuficiencia de Oferta} & CEL registrados < 70\% de la obligación & Hasta 50\% por dos años \\
    \textbf{Precio Elevado} & Precio implícito > 60 UDIs & Hasta 50\% por dos años \\
    \textbf{Condiciones Normales} & No se cumplen las anteriores & Hasta 25\% por dos años \\
    \bottomrule
  \end{tabularx}
\end{tabladoradoCorto}

\section{Criterios para Imposición de Sanciones}

\subsection{Principios Generales}

La resolución establece criterios orientadores que faciliten la determinación del monto de sanción, considerando:

\textbf{1. Gravedad de la infracción}

\textbf{2. Capacidad económica del infractor}

\textbf{3. Reincidencia}

\textbf{4. Comisión del hecho que la motiva}

\textbf{5. Acciones tomadas para corregirlo}

\subsection{Metodología de Cálculo}

\textbf{Fórmula base:}

La multa se aplicará por cada megawatt-hora de incumplimiento en la adquisición de CEL, determinado como:

\textbf{Porcentaje de Incumplimiento = } $\frac{\text{Obligación de Energías Limpias} - \text{CEL Liquidados}}{\text{Obligación de Energías Limpias}} \times 100$

\subsection{Factores Agravantes}

\textbf{Diferimiento previo:} La resolución considera que diferir el cumplimiento y luego incumplir constituye un agravante, señalando posible intencionalidad.

\textbf{Grado de incumplimiento:} La gravedad aumenta proporcionalmente con el porcentaje de incumplimiento.

\section{Matriz de Sanciones}

\subsection{Tabla de Multas por Incumplimiento}

\begin{tabladoradoLargo}
    \tiny
    \begin{xltabular}{\textwidth}{|l|X|X|X|X|}
    \caption{Matriz para Determinación de Multas por Incumplimiento CEL} \\
    \toprule
    \rowcolor{gobmxDorado} \multicolumn{5}{|c|}{\encabezadodorado{Días de Salario Mínimo por MWh Incumplido}} \\
    \midrule
    \rowcolor{gobmxDorado} \encabezadodorado{Porcentaje de Incumplimiento} & \encabezadodorado{Sin Diferimiento - Primera Vez} & \encabezadodorado{Sin Diferimiento - Reincidencia} & \encabezadodorado{Con Diferimiento - Primera Vez} & \encabezadodorado{Con Diferimiento - Reincidencia} \\
    \midrule
    \endhead
    
    \textbf{> 0\% y ≤ 25\%} & 6 & 12 & 8 & 16 \\ \hline
    \textbf{> 25\% y ≤ 50\%} & 8 & 16 & 10 & 20 \\ \hline
    \textbf{> 50\% y ≤ 75\%} & 10 & 20 & 12 & 24 \\ \hline
    \textbf{> 75\% y ≤ 100\%} & 12 & 24 & 14 & 28 \\ \hline
    \textbf{Tercera vez o contumacia} & 18-36 & - & 24-42 & - \\ \hline

    \bottomrule
    \end{xltabular}
\end{tabladoradoLargo}

\subsection{Aplicación de la Matriz}

\textbf{Pasos para determinar la sanción:}

\textbf{1.} Verificar si el participante difirió o no el cumplimiento de obligaciones

\textbf{2.} Determinar el grado de incumplimiento (porcentaje)

\textbf{3.} Identificar si es primera vez, reincidencia o contumacia

\textbf{4.} Aplicar el número indicativo de días de salario mínimo

\textbf{5.} Considerar otros elementos que puedan modificar la sanción

\section{Procedimiento Sancionador}

\subsection{Notificación Previa}

Antes de iniciar el procedimiento administrativo sancionador, la Comisión:

\textbf{• Notificará} al Participante Obligado la falta de cumplimiento

\textbf{• Otorgará} un plazo de tres meses improrrogables para regularizarse

\textbf{• Exonerará} de sanción si se regulariza en el plazo otorgado

\subsection{Proceso Administrativo}

\begin{tabladoradoCorto}
  \caption{Etapas del Procedimiento Sancionador}
  \begin{tabularx}{\textwidth}{L{0.25\textwidth} X X}
    \toprule
    \rowcolor{gobmxDorado} \encabezadodorado{Etapa} & \encabezadodorado{Descripción} & \encabezadodorado{Marco Legal} \\
    \midrule
    \textbf{Notificación} & Aviso de incumplimiento y plazo de regularización & Resolución RES/248/2016 \\
    \textbf{Procedimiento} & Aplicación de la LFPA & Artículo 169 LIE \\
    \textbf{Resolución} & Determinación de sanción & Artículos 165-166 LIE \\
    \textbf{Cobro} & Facturación por CENACE & Artículo 166 LIE \\
    \bottomrule
  \end{tabularx}
\end{tabladoradoCorto}

\subsection{Cobro de Multas}

\textbf{Mecanismo de cobro:}

\textbf{• Instrucción:} La CRE instruye a CENACE

\textbf{• Proceso:} A través del sistema de facturación del MEM

\textbf{• Destino:} Fondo de Servicio Universal Eléctrico

\textbf{• Reporte:} CENACE informa sobre el pago en 15 días

\section{Definiciones de Reincidencia y Contumacia}

\subsection{Reincidencia}

Existe reincidencia cuando:

\textbf{• Se ha impuesto} una sanción por incumplimiento de CEL

\textbf{• La sanción ha quedado} firme (administrativa o judicialmente)

\textbf{• El infractor incurre} en nuevos incumplimientos

\textbf{Sanción:} Doble de la que se hubiere aplicado la primera vez

\subsection{Contumacia}

Se presenta contumacia cuando:

\textbf{• El infractor se resiste} al cumplimiento de la obligación legalmente exigida

\textbf{Sanción:} Triple de la que se hubiere aplicado la primera vez, además de suspensión temporal o definitiva del servicio

\section{Principios del Régimen Sancionador}

\subsection{Proporcionalidad}

Las sanciones deben ser acordes a:

\textbf{• Gravedad} de la infracción

\textbf{• Capacidad económica} del infractor

\textbf{• Daños} producidos o que puedan producirse

\textbf{• Carácter intencional} o no de la acción

\subsection{Finalidad del Sistema}

\textbf{Objetivo principal:} El pago de la sanción NO exime del cumplimiento de la obligación de adquirir CEL.

\textbf{Justificación:} Evitar que el incumplimiento ponga en riesgo la diversificación de la matriz energética y la promoción de energías limpias.

\section{Aspectos Procedimentales}

\subsection{Ley Aplicable}

Para la imposición de sanciones se aplicará la Ley Federal de Procedimiento Administrativo (LFPA), conforme al artículo 169 de la LIE.

\subsection{Medios de Impugnación}

\textbf{Recurso disponible:} Juicio de amparo indirecto

\textbf{Fundamento:} Artículo 27 de la Ley de Órganos Reguladores

\textbf{Nota:} Se abrogó el recurso de reconsideración previsto en la ley anterior

\subsection{Registro de la Resolución}

\textbf{Número de registro:} RES/248/2016

\textbf{Fundamento:} Artículos 11, 22 fracción XXVI inciso a) y 25 fracción X de la LORCME

\section{Consideraciones Especiales}

\subsection{Capacidad Económica}

La determinación de la capacidad económica se basará en:

\textbf{Para Usuarios Finales:}
\textbf{• Consumo} de energía eléctrica realizado

\textbf{Para Usuarios Calificados y Suministradores:}
\textbf{• Adquisición} de energía para atender necesidades de clientes

\subsection{Cámara de Compensación}

La LTE prevé el desarrollo de una cámara de compensación que facilite:

\textbf{• Participación} en subastas de usuarios calificados

\textbf{• Realización} de subastas propias

\textbf{• Adquisición} de contratos de cobertura de CEL

\section{Vigencia y Transitorios}

\subsection{Entrada en Vigor}

La resolución entró en vigor al día siguiente de su publicación en el DOF (28 de abril de 2016).

\subsection{Aplicación Temporal}

\textbf{Aplicable a:} Incumplimientos posteriores a su entrada en vigor

\textbf{Procedimientos en curso:} Se rigen por la normatividad anterior

\section{Impacto y Objetivos}

\subsection{Objetivos de la Resolución}

\textbf{1. Certeza jurídica:} Criterios objetivos para determinación de sanciones

\textbf{2. Proporcionalidad:} Multas acordes a la gravedad del incumplimiento

\textbf{3. Disuasión:} Desincentivar el incumplimiento de obligaciones

\textbf{4. Cumplimiento:} Mantener la obligación pese al pago de multa

\subsection{Beneficios Esperados}

\textbf{• Transparencia} en el proceso sancionador

\textbf{• Predictibilidad} para los participantes del mercado

\textbf{• Fortalecimiento} del sistema de CEL

\textbf{• Promoción} efectiva de energías limpias

\section{Conclusiones}

La Resolución RES/248/2016 establece un marco regulatorio claro y objetivo para la imposición de sanciones por incumplimiento de obligaciones en materia de CEL. Su diseño busca equilibrar la necesidad de hacer cumplir las obligaciones de energías limpias con principios de proporcionalidad y debido proceso.

\textbf{Elementos clave:}

\textbf{• Matriz objetiva} de determinación de multas

\textbf{• Procedimiento} con oportunidad de regularización

\textbf{• Principio} de que la sanción no exime del cumplimiento

\textbf{• Consideración} de factores agravantes y atenuantes

Esta resolución constituye un instrumento fundamental para el funcionamiento efectivo del sistema de CEL y el cumplimiento de los objetivos de la transición energética nacional.

\section*{Disposiciones Finales}

\textbf{Primera.} La presente resolución se inscribe bajo el número RES/248/2016 en el registro correspondiente.

\textbf{Segunda.} El expediente respectivo se encuentra disponible para consulta en las oficinas de la Comisión Reguladora de Energía.

\textbf{Tercera.} Esta resolución sólo podrá impugnarse a través del juicio de amparo indirecto.

\textbf{Cuarta.} Los ingresos percibidos por el cobro de sanciones se destinarán al Fondo de Servicio Universal Eléctrico.

\end{document}